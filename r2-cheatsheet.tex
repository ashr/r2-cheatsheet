\documentclass[a4paper,landscape]{article}
\usepackage[bookmarks=true,colorlinks=true,linkcolor=black,citecolor=black,filecolor=black,urlcolor=black]{hyperref}
\usepackage{multicol}
\usepackage{geometry}
\geometry{top=1cm,left=1cm,right=1cm,bottom=1cm}
\pagestyle{empty}

% Redefine section commands to use less space
\makeatletter
\renewcommand{\section}{\@startsection{section}{1}{0mm}%
                                {-1ex plus -.5ex minus -.2ex}%
                                {0.5ex plus .2ex}%x
                                {\normalfont\large\bfseries}}
\renewcommand{\subsection}{\@startsection{subsection}{2}{0mm}%
                                {-1explus -.5ex minus -.2ex}%
                                {0.5ex plus .2ex}%
                                {\normalfont\normalsize\bfseries}}
\renewcommand{\subsubsection}{\@startsection{subsubsection}{3}{0mm}%
                                {-1ex plus -.5ex minus -.2ex}%
                                {1ex plus .2ex}%
                                {\normalfont\small\bfseries}}
\makeatother

\setcounter{secnumdepth}{0} % Don't print section numbers
\setlength{\parindent}{0pt}
\setlength{\parskip}{0pt plus 0.5ex}
\begin{document}
%\raggedright
\footnotesize
\begin{multicols*}{3}
% multicol parameters
% These lengths are set only within the two main columns
%\setlength{\columnseprule}{0.25pt}
\setlength{\premulticols}{1pt}
\setlength{\postmulticols}{1pt}
\setlength{\multicolsep}{1pt}
\setlength{\columnsep}{2pt}

\begin{center}
     \Large{\textbf{Reversing with Radare2}} \\
\end{center}

\section{Starting Radare}
The basic usage is \texttt{radare2} \textit{executable} (on some systems you can
use \texttt{r2} instead of \texttt{radare2}); if you want to run radare2 without
opening any file, you can use \texttt{--} instead of an executable name.

Some command-line options are:

\begin{tabular}{@{}ll@{}}
\texttt{-d} \textit{file}$|$\textit{pid} & debug executable \textit{file} or process \textit{pid} \\
\texttt{-A}	& analyze all referenced code (\texttt{aaa} command) \\
\texttt{-R} \textit{profile}\texttt{.rr2} & specifies \texttt{rarun2} profile (same as \\ & \texttt{-e dbg.profile=}\textit{profile}\texttt{.rr2}) \\
\texttt{-w}	& open file in write mode \\
\texttt{-p} \textit{prj} & use project \textit{prj} \\
\texttt{-p} & list projects \\
\texttt{-h} & show help message (\texttt{-hh} the verbose one)
\end{tabular}

Example: \texttt{r2 -dA /bin/ls}

\section{General information}
The command \texttt{?} prints the help. Command names are hierarchically defined; for instance, all \textbf{p}rinting commands start with \texttt{p}. So, to understand what a command does, you can append \texttt{?} to a \emph{prefix} of such a command; for instance, to learn what \texttt{pdf} does, you can first try \texttt{pd?}, then the more general \texttt{p?}.

Single-line comments can be entered using \texttt{\#}; e.g. \texttt{s \# where R we?}.

Command \texttt{?} can also be used to evaluate an expression and print its result in various format; e.g. \texttt{? 5 * 8 + 2} (note the space between \texttt{?} and the expression).  There are also some special \texttt{\$}-variables
(list all of them with: \texttt{?\$?}):

\begin{tabular}{@{}ll@{}}
\texttt{\$\$} & current virtual seek \\
\texttt{\$b} & 	block size \\
\end{tabular}

Where an address \textit{addx} is expected, you can provide any expression that evaluates to an address,
e.g. a function name or a register name. In this cheatsheet we sometimes use \textit{fn-name}, instead of \textit{addx}, to emphasize that the argument is supposed to be a function starting address. As default address is (usually?) used the current seek: \texttt{\$\$}.

All commands that:
\begin{itemize}
\item accept an optional size (e.g. \texttt{pd}), use the current block size by default (see: \texttt{b})
\item accept an optional address (e.g., \texttt{pdf}), use the current position by default (see: \texttt{s})
\end{itemize}

\subsection{Internal grep-like filtering}
You can filter command output by appending \texttt{\textasciitilde{}}[\texttt{!}]\textit{str}, to display only rows [not] containing string \textit{str}; e.g. \texttt{pdf\textasciitilde{}rdx} and \texttt{pdf\textasciitilde{}!rdx}. You can further filter by appending

\begin{tabular}{@{}ll@{}}
\texttt{:}\textit{r}	& to display row \textit{r} ($0\le r < \#\mathit{rows}$ or, backwards \\ & with: $-\#\mathit{rows}\le r \le -1)$ \\
\texttt{[}\textit{c}\texttt{]}	& to display column \textit{c} ($0\le c < \#\mathit{cols}$) \\
\texttt{:r[}\textit{c}\texttt{]}	& to display column \textit{c} of row $r$
\end{tabular}

Examples: \texttt{afl\textasciitilde{}[0]}, \texttt{afl\textasciitilde{}malloc[0]}, \texttt{pdf\textasciitilde{}:2} and \texttt{pdf\textasciitilde{}mov:2}

\subsection{Shell interaction}
Command output can be redirected to a file by appending \texttt{>}\textit{filename}
or piped to an external command with \texttt{|}\textit{progname} [\textit{args}].
Examples: \texttt{afl > all\_functions} and \texttt{afl | wc -l}.

External commands can be run with \texttt{!!}\textit{progname} [\textit{args}].
Note: if a command starts with a single \texttt{!}, the rest of the string is passed to currently loaded IO plugin (only if no plugin can handle the command, it is passed to the shell).

The output of external programs can be used as arguments for internal commands by using back-ticks to enclose the invocation of external commands; e.g. \texttt{pdf `echo 3` @ `echo entry0`}.

\subsection{Python scripting}
Assuming that Python extension has been installed (\texttt{\#!} lists installed extensions)
an, interactive Python interpreter can be spawned with \texttt{\#!python} and a script can
be run with \texttt{\#!python }\textit{script-filename}.

Inside the spawned interpreter \texttt{r2} is an \emph{r2pipe} object that can be used to interact with
the same instance of Radare, by invoking method \texttt{cmd}; e.g. \texttt{print(r2.cmd('pdf @ entry0'))}.

In a script the same behaviour can be obtained by \texttt{import}ing \texttt{r2pipe} and inizializing \texttt{r2} with \texttt{r2pipe.open("\#!pipe")}.

You can make most Radare2 commands output in JSON format by appending a \texttt{j}; e.g. \texttt{pdfj} (instead of \texttt{pdf}).

Method \texttt{cmdj} can de-serialize JSON output into Python objects; e.g. \texttt{f = r2.cmdj('pdfj @ entry0')} \\ \texttt{print f['name'], f['addr'], f['ops'][0]['opcode']}

\subsection{Configuration}
\begin{tabular}{@{}ll@{}}
\texttt{e??} & list all variable names and descriptions \\
\texttt{e?}[\texttt{?}] \textit{var-name} & show description of \textit{var-name}\\
\texttt{e} \textit{var-name} & show the value of \textit{var-name} \\
\texttt{e} & show the value of all variables \\
\texttt{eco} \textit{theme-name} & select theme; eg. \texttt{eco solarized} \\
\texttt{eco} & list available themes \\
\texttt{b} & display current block size \\
\texttt{b} \textit{size} & set block size \\
\texttt{env} [\textit{name} [\texttt{=}\textit{value}]] & get/set environment variables \\
\end{tabular}
\subsubsection{Some variables}
\begin{tabular}{@{}ll@{}}
\texttt{asm.pseudo} & enable pseudo-code syntax \\ & (in visual mode, toggle with: \texttt{\$}) \\
\texttt{asm.bytes} & display bytes of each instruction \\
\texttt{asm.cmtright} & comments at right of disassembly if they fit \\
\texttt{asm.emu} & run ESIL emulation analysis on disasm \\
\texttt{asm.demangle} & Show demangled symbols in disasm \\
\\
\texttt{bin.demangle} & Import demangled symbols from RBin \\
\\
\texttt{cmd.stack} & command to display the stack in visual \\ & debug mode (Eg: \texttt{px 32}) \\
\\
\texttt{dbg.follow.child} & continue tracing the child process on fork \\
\\
\texttt{io.cache} & enable cache for IO changes \\ & (AKA non-persistent write-mode) \\
\\
\texttt{scr.utf8} & show nice UTF-8 chars instead of ANSI \\ & (Windows: switch code-page with \texttt{chcp 65001})\\
\end{tabular}
\subsubsection{Example: my \texttt{\textasciitilde{}/.radare2rc}}
\begin{verbatim}
e asm.bytes=0
e scr.utf8=true
e asm.cmtright=true
e cmd.stack=px 32
eco solarized
\end{verbatim}

\section{Searching: \texttt{/}}
\begin{tabular}{@{}ll@{}}
\texttt{/ \textit{str}} & search for string \textit{str} \\
\texttt{/x \textit{hstr}} & search for hex-string \textit{hstr} \\
\texttt{/a \textit{asm-instr}} & assemble instruction and search for its bytes \\
\texttt{/R }\textit{opcode} & find ROP gadgets containing opcode; \\ & see: \url{http://radare.today/posts/ropnroll/} \\
\end{tabular}

\emph{a lot} of other commands\ldots TODO!

Also: \texttt{e??search} for options

\section{Seeking: \texttt{s}}
\begin{tabular}{@{}ll@{}}
\texttt{s} & print current position/address \\
\texttt{s} \textit{addx} & seek to \textit{addx} \\
\texttt{s+} \textit{n} & seek \textit{n} bytes forward \\
\texttt{s++} & seek block-size bytes forward \\
\texttt{s-} \textit{n} & seek \textit{n} bytes backward \\
\texttt{s--} & seek block-size bytes backward \\
\texttt{s-} & undo seek \\
\texttt{s+} & redo seek \\
\texttt{s=} & list seek history \\
\texttt{s*} & list seek history as r2-commands \\
\end{tabular}

\section{Analysis (functions and syscalls): \texttt{a}}
\begin{tabular}{@{}ll@{}}
\texttt{aaa} & analyze (\texttt{aa}) and auto-name all functions \\
\texttt{afl} & list functions \\
\texttt{afll} & list functions with details \\
\texttt{afi} \textit{fn-name} & show verbose info for \textit{fn-name} \\
\texttt{afn} \textit{new-name} \textit{addx} & name function at address \textit{addx} \\
\texttt{afn} \textit{new-name} \textit{old-name} & rename function \\
\texttt{asl} & list syscalls \\
\texttt{asl} \textit{name} & display syscall-number for \textit{name} \\
\texttt{asl} \textit{n} & display name of syscall number \textit{n} \\
\texttt{afvd} \textit{var-name} & output r2 command for displaying the \\ & address and value of arg/local \textit{var-name} \\
\texttt{.afvd} \textit{var-name} & display address and value of \textit{var-name} \\
\texttt{afvn} \textit{name} \textit{new-name} & rename argument/local variable \\
\texttt{afvt} \textit{name} \textit{type} & change type for given argument/local \\
\texttt{axt} \textit{addx} & find data/code references to \textit{addx} \\
\end{tabular}
\subsection{Graphviz/graph code: \texttt{ag}}
\begin{tabular}{@{}ll@{}}
\texttt{ag} \textit{addr} & output graphviz code (BB at \textit{addr} and children) \\ &  E.g. view the function graph with: \texttt{ag \$\$ | xdot -} \\
\texttt{agc} \textit{addr} & callgraph of function at \textit{addx} \\
\texttt{agC} & full program callgraph \\
\end{tabular}

\section{Information: \texttt{i}}
\begin{tabular}{@{}ll@{}}
\texttt{i} & show info of current file \\
\texttt{ie} & entrypoint \\
\texttt{iz} & strings in data sections \\
\texttt{izz} & strings in the whole binary \\
\texttt{ii} & imports \\
\texttt{iS} & sections \\
\end{tabular}

\section{Printing: \texttt{p}}
\begin{tabular}{@{}ll@{}}
\texttt{ps} [\texttt{@} \textit{addx}]& print C-string at \textit{addx} (or current position) \\
\texttt{pxr} [\textit{n}] [\texttt{@} \textit{addx}] & print \textit{n} bytes (or block-size), as words, with \\ & references to flags and code (telescoping) at \\ & \textit{addx} (or current position)\\
\texttt{px} [\textit{n}] [\texttt{@} \textit{addx}] & hexdump \\
\texttt{pxh} \ldots & hexdump half-words (16 bits) \\
\texttt{pxw} \ldots & hexdump words (32 bits) \\
\texttt{pxq} \ldots & hexdump quad-words (64 bits) \\
\texttt{pxl} [\textit{n}] [\texttt{@} \textit{addx}] & display \textit{n} rows of hexdump \\
\texttt{px/}\textit{fmt} [\texttt{@} \textit{addx}] & gdb-style printing \textit{fmt} (in gdb see: \texttt{help x} \\ & from r2: \texttt{!!gdb -q -ex 'help x' -ex quit}) \\
\texttt{pd} [\textit{n}] [\texttt{@} \textit{addx}] & disassemble \textit{n} instructions \\
\texttt{pD} [\textit{n}] [\texttt{@} \textit{addx}] & disassemble \textit{n} bytes \\
\texttt{pd} \textit{-n}  [\texttt{@} \textit{addx}] & disassemble \textit{n} instructions backwards \\
\texttt{pdf} [\texttt{@} \textit{fn-name}] & disassemble function \textit{fn-name} \\
\texttt{pdc} [\texttt{@} \textit{fn-name}] & pseudo-disassemble in C-like syntax
\end{tabular}

\section{Debugging: \texttt{d}}
\begin{tabular}{@{}ll@{}}
\texttt{?d} \textit{opcode} & description of \textit{opcode} (eg. \texttt{?d jle}) \\ & BUG (?): this doesn't work on Windows \\
\texttt{dc} & continue (or start) execution \\
\texttt{dcu} \textit{addx} & continue until \textit{addx} is reached \\
\texttt{dcs} [\textit{name}] & continue until the next syscall (named \textit{name}, \\ & if specified) \\
\texttt{dcr} & continue until ret (uses step over) \\
\texttt{dr=} & show general-purpose regs and their values \\
\texttt{dro} & show previous (old) values of registers \\
\texttt{drr} & show register references (telescoping) \\
\texttt{dr} \textit{reg-name} \texttt{=} \textit{value} & set register value \\
\texttt{drt} & list register types \\
\texttt{drt} \textit{type} & list registers of type \textit{type} and their values \\
\texttt{db} & list breakpoints \\
\texttt{db }\textit{addx} & add breakpoint \\
\texttt{db -}\textit{addx} & remove breakpoint \\
\texttt{doo \textit{args}} & (re)start debugging \\
\texttt{ood} & synonym for \texttt{doo} \\
\texttt{ds} & step into \\
\texttt{dso} & step over \\
\texttt{dbt} & display backtrace \\
\texttt{drx} & hardware breakpoints \\
\texttt{dm} & list memory maps; the asterisk shows where \\ & the current offset is \\
\texttt{dmp} & change page permissions (see: \texttt{dmp?}) \\
\end{tabular}

\section{Types: \texttt{t}}
\begin{tabular}{@{}ll@{}}
\texttt{"td\ }\textit{C-type-def}\texttt{"} & define a new type \\
\texttt{t} \textit{t-name} & show type \textit{t-name} in \texttt{pf} syntax \\
\texttt{.t }\textit{t-name} \texttt{@} \textit{addx} & display the value (of type \textit{t-name}) at \textit{addx} \\
\texttt{t} & list (base?) types \\
\texttt{te} & list enums \\
\texttt{ts} & list structs \\
\texttt{tu} & list unions \\
\texttt{to} \textit{file} & parse type information from C header file \\
\texttt{tl} \textit{t-name} & link \textit{t-name} to current address \\
\texttt{tl} \textit{t-name} \texttt{=} \textit{addx} & link \textit{t-name} to address \textit{addx} \\
\texttt{tl} & list all links in readable format \\
\end{tabular}

\section{Visual mode: \texttt{V}}
Command \texttt{V} enters \emph{visual mode}.

\begin{tabular}{@{}ll@{}}
\texttt{q} & exit visual-mode \\
\texttt{c} & cursor-mode, \textit{tab} switches among stack/regs/disassembly \\
\texttt{:} & execute a normal-mode command; e.g. \texttt{:dm} \\
\texttt{p} and \texttt{P} & rotate forward/backward print modes \\
\texttt{/}\textit{str} & highlight occurences of string \textit{str} \\
\texttt{\$} & toggle pseudo-syntax \\
\texttt{O} & toggle ESIL-asm \\
\texttt{;} & add/remove comments (to current offset) \\
\texttt{x} & browse xrefs-to current offset \\
\texttt{X} & browse xrefs-from current function \\
\texttt{\_} & browse flags \\
\texttt{d} & define \texttt{f}unction, \texttt{e}nd-function, \texttt{r}ename, \ldots \\
\texttt{V} & enter block-graph viewer \\
\texttt{A} & enter visual-assembler \\
\end{tabular}
\subsection{Seeking (in Visual Mode)}
\begin{tabular}{@{}ll@{}}
\texttt{.} & seeks to program counter \\
\textit{Enter} & on jump/call instructions, follow target address \\
\texttt{u} & undo \\
\texttt{U} & redo \\
\texttt{o} & go/seek to given offset \\
\textit{d} (a digit) & jump to the target marked \texttt{[}\textit{d}\texttt{]} \\
\texttt{m}\textit{l} (a letter) & mark the spot with letter \textit{l} \\
\texttt{'}\textit{l} & jump to mark \textit{l} \\
\texttt{n} & jump to next function \\
\texttt{N} & jump to previous function
\end{tabular}
\subsection{Debugging (in Visual Mode)}
\begin{tabular}{@{}ll@{}}
\texttt{b} or \texttt{F2} & toggle breakpoint \\
\texttt{F4} & run to cursor \\
\texttt{s} or \texttt{F7} & step-into \\
\texttt{S} or \texttt{F8} & step-over \\
\texttt{F9} & continue \\
\end{tabular}

\section{Flags: \texttt{f}}
TODO

\section{Comments: \texttt{C}}
TODO

\section{Writing: \texttt{w}}
\begin{tabular}{@{}ll@{}}
\texttt{wa} \textit{asm-instr} & assemble and write opcodes; for more instructions \\ &
	the whole command must be quoted: \\ &
\texttt{"wa} \textit{asm-instr$_1$}\texttt{;} \textit{asm-instr$_2$}\texttt{;} \ldots\texttt{"} \\
\end{tabular}

\section{Projects: \texttt{P}}
TODO

\section{Running in different environments: rarun2}
\texttt{rarun2} is used as a launcher for running programs with different environment,
arguments, permissions, directories and overridden default file-descriptors.
Usage:

\texttt{rarun2} [\texttt{-t}$|$\textit{script-name}\texttt{.rr2}] [\textit{directives}] [\texttt{--}] [\textit{prog-name}] [\textit{args}]

\texttt{rarun2 -t} shows the terminal name, say $\alpha$, and wait for a connection from another process. For instance, from another terminal, you can execute \texttt{rarun2 stdio=$\alpha$ program=/bin/sh} (use \texttt{stdin}/\texttt{stdout} to redirect one stream only).

rarun2 supports \emph{a lot} of directives, see the man page.


\rule{1.0\linewidth}{0.25pt}
\scriptsize
Copyright \copyright 2017 by zxgio

This cheat-sheet may be freely distributed under the terms of the GNU
General Public License; the latest version can be found at: \\
\url{https://github.com/zxgio/r2-cheatsheet/}
\end{multicols*}
\end{document}
